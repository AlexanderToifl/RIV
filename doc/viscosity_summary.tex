\documentclass[a4paper, 10pt, 
               numbers=noenddot, toc=graduated,
               headsepline=true, footsepline=true,
               twoside=false, titlepage=true, 
               bibliography=totoc]{scrartcl}



\usepackage[a4paper, top = 2.5cm,
  	bottom = 2.0cm,
    left   = 2.0cm,
    right  = 2.0cm]{geometry}

\usepackage{graphicx}
\usepackage{amsmath} %math package
\usepackage{amssymb}
\usepackage[font={footnotesize}, bf]{caption}
\renewcommand{\arraystretch}{1.2}

\usepackage{longtable}
\usepackage{lscape}
%\usepackage{siunitx}	
%\usepackage{citesort} 

\usepackage[toc,page]{appendix}
\setlength\parindent{0pt}
\usepackage{subcaption}
\usepackage{cite} 
%\usepackage{float} 
\usepackage[utf8x]{inputenc}		% input encoding
\usepackage[ngerman]{babel} 
\captionsetup{format=plain}
\addto\captionsenglish{\renewcommand{\figurename}{Fig.}}
\addto\captionsenglish{\renewcommand{\tablename}{Tab.}}
\usepackage[T1]{fontenc}                % mathmode for font Palatino
%\usepackage[plainfootsepline ,autooneside]{scrpage2}	% header and footer for komascript
\usepackage{hyperref}
\usepackage[procnames]{listings}

\usepackage{url}
\usepackage{titling}
\usepackage{multirow}
 

\usepackage{floatrow}
% Table float box with bottom caption, box width adjusted to content
\newfloatcommand{capbtabbox}{table}[][\FBwidth]


\usepackage{array}

\makeatletter
\newcommand{\thickhline}{%
    \noalign {\ifnum 0=`}\fi \hrule height 1pt
    \futurelet \reserved@a \@xhline
}
\newcolumntype{"}{@{\hskip\tabcolsep\vrule width 1pt\hskip\tabcolsep}}
\makeatother

%\usepackage{fancyhdr}
%\pagestyle{fancy}
%\fancyhf{}
%\fancyheadoffset{0 cm}

%\rhead{\nouppercase\rightmark}
%\lhead{\thepage}

\newif\ifinternal

\internalfalse

\newcommand*\mean[1]{\overline{#1}}

\begin{document}

\title{Bestimmung der strahlungsinduzierten Viskosit{\"a}t mit Hilfe von LAMMPS}
\author{Christian Schleich, Alexander Toifl}

\begin{center}
\huge{\thetitle} \\
\small{\theauthor, \today}
\end{center}

%\tableofcontents
\hspace{1 cm}


\section{Strahlungsinduzierte und dynamische Viskosität}
\begin{itemize}
	 \item Definition der Strahlungsinduzierten Fluidität $H = \frac 1 {\eta \dot{\phi}}$ ($\eta$ ... Viskosität, $\dot{\phi}$ ... 'flux') \cite{Mayr2003}
\end{itemize}


\section{MD Methoden zur Bestimmung der Viskosität}

\subsection{Non-equilibrium MD (NEMD) Simulationen}

System wird verformt (Scherung) und die resultierende Spannung wird ermittelt. 


	\subsubsection{SLLOD}
		\begin{itemize}
		 	\item Zusammenfassende Beschreibung \cite{Tenney2010} 
		 	\item Original Paper, indem SLLOD Dynamik vorgestellt werden \cite{Evans1984}
		 	\item LAMMPS: Simulationsbox wird geschert. Kombination von fix nvt/sllod und fix deform.
		\end{itemize}
		
	\subsubsection{Moving Walls}
	    \begin{itemize}
		 	\item LAMMPS: 'walls' werden definiert, eine ist ortfest, die zweite bewegt sich. Damit wird Scherung erreicht
		\end{itemize}

\subsection{Reverse Non-Equlibirium MD  (rNEMD) Simulationen}
	\begin{itemize}
		 \item Zusammenfassende Beschreibung \cite{Tenney2010} 
		 \item LAMMPS MD Simulationen \cite{Tenney2010} 
	\end{itemize}



\subsection{Autokorrelation des Drucktensors (Green-Kubo)}
	\begin{itemize}
		 \item Zusammenfassende Beschreibung \cite{Tenney2010}
		 \item Mathematische Formulierung  \cite{Tenney2010} (in Abschnitt EMD, eher minimal gehalten), \cite{Kirova2015} (etwas ausführlicher)
		 \item Bsp in LAMMPS verweist auf \cite{Daivis1994}
	\end{itemize}

\subsection{Einstein Formulierung der Green-Kubo Methode}
	\begin{itemize}
		 \item Zusammenfassende Beschreibung \cite{Tenney2010}
		 \item Mathematische Formulierung  \cite{Tenney2010} (in Abschnitt EMD)
	\end{itemize}

\section{Experimentelle Befunde}

\subsection{Stress und Viskositätsbetrachtungen}
	\begin{itemize}
		\item Betrachtungen für $SiO_2$ in \cite{Snoeks2000}
	\end{itemize}




\bibliographystyle{Citation/IEEEbib-c.bst}
\bibliography{Citation/MolecularDynamics}


\end{document}  
