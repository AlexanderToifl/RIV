\documentclass[a4paper, 10pt, 
               numbers=noenddot, toc=graduated,
               headsepline=true, footsepline=true,
               twoside=false, titlepage=true, 
               bibliography=totoc]{scrartcl}



\usepackage[a4paper, top = 2.5cm,
  	bottom = 2.0cm,
    left   = 2.0cm,
    right  = 2.0cm]{geometry}

\usepackage{graphicx}
\usepackage{amsmath} %math package
\usepackage{amssymb}
\usepackage[font={footnotesize}, bf]{caption}
\renewcommand{\arraystretch}{1.2}

\usepackage{longtable}
\usepackage{lscape}
\usepackage{siunitx}	
%\usepackage{citesort} 

\usepackage[toc,page]{appendix}
\setlength\parindent{0pt}
\usepackage{subcaption}
\usepackage{cite} 
%\usepackage{float} 
\usepackage[utf8x]{inputenc}		% input encoding
\usepackage[ngerman]{babel} 
\captionsetup{format=plain}
\addto\captionsenglish{\renewcommand{\figurename}{Fig.}}
\addto\captionsenglish{\renewcommand{\tablename}{Tab.}}
\usepackage[T1]{fontenc}                % mathmode for font Palatino
%\usepackage[plainfootsepline ,autooneside]{scrpage2}	% header and footer for komascript
\usepackage{hyperref}
\usepackage[procnames]{listings}

\usepackage{url}
\usepackage{titling}
\usepackage{multirow}
 

\usepackage{floatrow}
% Table float box with bottom caption, box width adjusted to content
\newfloatcommand{capbtabbox}{table}[][\FBwidth]


\usepackage{array}

\makeatletter
\newcommand{\thickhline}{%
    \noalign {\ifnum 0=`}\fi \hrule height 1pt
    \futurelet \reserved@a \@xhline
}
\newcolumntype{"}{@{\hskip\tabcolsep\vrule width 1pt\hskip\tabcolsep}}
\makeatother

\newcommand{\refeqn}[1]  {Glg.~\ref{#1}}  

%\usepackage{fancyhdr}
%\pagestyle{fancy}
%\fancyhf{}
%\fancyheadoffset{0 cm}

%\rhead{\nouppercase\rightmark}
%\lhead{\thepage}

\newif\ifinternal

\internalfalse

\newcommand*\mean[1]{\overline{#1}}

\begin{document}

\title{Bestimmung der strahlungsinduzierten Viskosit{\"a}t mit Hilfe von LAMMPS}
\author{Christian Schleich, Alexander Toifl}

\begin{center}
\huge{\thetitle} \\
\small{\theauthor, \today}
\end{center}

%\tableofcontents
\hspace{1 cm}


\section{Strahlungsinduzierte und dynamische Viskosität}

Die im betrachteten Paper \cite{Mayr2003} beschreibene Methode wird in \cite{hobler2017hpm} schön zusammengefasst.

\begin{itemize}
	 \item Definition der Strahlungsinduzierten Fluidität 
	 	\begin{equation}
	  		H = \frac 1 {\eta \dot{\phi}} = \frac 1 {\eta^{'}}
		\end{equation}  
		$\eta$ ... Viskosität\\
		$\dot{\phi}$ ... 'flux'\\
		$\eta^{'}$ ... strahlungsind. Viskosität \cite{Mayr2003}
\end{itemize}


\section{MD Methoden zur Bestimmung der Viskosität}

Allgemeine Informationen zum Aufschmelzen und Abkühlen (Temperatur und Druck) mit MD - Simulationen (in Ge - Dünnschichten) zur Amorphisierung des Materials finden sich in \cite{Mayr2005}. Außerdem werden auch Werte für die strahlungsinduzierte Viskosität in Abhängigkeit der Defektdichte dpa angegeben (Methode?). Ebenfalls wird dies in \cite{Edler2007} beschrieben, wobei hier aber nur mech. Stressbetrachtungen gemacht werden. In \cite{Lehnert2017} wird die in \cite{Mayr2003} erwähnte Methode auf a-Ge-Si Legierungen angewandt um mithilfe der Fluidität die Formierung von porösen Strukturen zu begründen.


\subsection{Non-equilibrium MD (NEMD) Simulationen}

System wird verformt (Scherung) und die resultierende Spannung wird ermittelt. 


	\subsubsection{SLLOD}
		\begin{itemize}
		 	\item Original Paper, indem SLLOD Dynamik vorgestellt werden \cite{Evans1984}, welche für alle homogenen Körper gilt.
		 	\item Dem System wird eine Schergeschwindigkeit (shear rate) $\dot{\gamma}$ aufgezwungen. Als Folge ergibt sich ein Impulsfluss, der über die Nicht-Diagonalelemente des Spannungs- oder Drucktensors gemessen wird\cite{Tenney2010}. Die Scherviskosität ergibt sich damit zu 
				\begin{equation}
					\eta\left(\dot{\gamma}\right) = - \frac{P_{ij}}{\dot{\gamma}}
					\label{equ:sllod}
				\end{equation}
 mit $P_{ij}$ den Nicht-Diagonalelementen des Stress - Tensors. i beschreibt die Flussrichtung und j die Normalrichtung darauf. 
		 	\item Problem: Druck variiert typischerweise sehr stark in MD Simulationen, daher ist die Konvergenz eher langsam\cite{Tenney2010}.
		 	\item LAMMPS: Mit der Verwendung des fix nvt/sllod und fix deform 
				\begin{itemize}
					\item fix nvt/sllod ist eine normale NVT simulation mit einem Nose/Hoover Thermostat, bei dem die Temperatur aus der kinetischen Energie der Teilchen abzüglich jener die durch die Scherung (die positionsabhängige "Fließgeschwindigkeit" jedes Atoms wird von der Gesamtgeschwindigkeit abgezogen) berechnet wird.
					\item fix deform verformt die Simulationsbox (zum Beispiel in xy Richtung mit konstanter Geschwindigkeit - siehe LAMMPS Beispiel in.nemd.2d)
					\item Um die Gleichung \ref{equ:sllod} anzuwenden wird der resultierende Impulsfluss (aus fix nvt/sllod - Beispiel pxy) durch die Konstante Verschiebungsgeschwindigkeit (hier in x - Richung gesetzt mit fix deform) dividiert durch die Boxlänge (hier y Abmessung der Box) dividiert. 
				\end{itemize}
		\end{itemize}
		
	\subsubsection{Moving Walls}
	    \begin{itemize}
		 	\item LAMMPS: 'walls' werden definiert, eine ist ortfest, die zweite bewegt sich. Damit wird Scherung erreicht
		\end{itemize}

\subsection{Reverse Non-Equlibirium MD  (rNEMD) Simulationen}
	\begin{itemize}
		 \item Zusammenfassende Beschreibung in \cite{Tenney2010}:
		  \item \textbf{Impulsfluss} wird vorgegeben und die \textbf{Schergeschwindigkeit} (Geschwindigkeitsprofil) wird gemessen.
		  \item Vorteil: Schergeschwindigkeit ist leichter aus MD Simulationen zu ermitteln
		  \item Simulationsbox wird in N kline Kästchen entlang der z-Achse eingeteilt
		  \item Vorgabe der Schergeschwindigkeit: \textbf{Periodischer} Tausch von Momenten (momenta swaps) 
		  		\begin{itemize}
			  		\item Ursprüngliche Implementierung: x-Komponente des Impulses des Atoms mit der negativsten Impulskompentente in Kästchen 1 $p_{x,n_1}$ wird mit der x-Komplente des Atomes mit der positivsten Impulskomponente im Kästchen $n = n_c = N/2 + 1$, $p_{x,n_c}$ getauscht.
			  		\item Lammps-Implementierung: Eine Zielgeschwindigkeit $\pm v_{x,t}$ wird definiert. Impuls der beiden Atome, die jeweils am knappsten am positiven/negativen Wert liegt, wird getauscht.
		  		\end{itemize}
		  \item Da Gesamtimpuls und kinetische Energie erhalten bleiben, ist prinzipiell kein Thermostat nötig
		  \item Die Tauschfrequenz bestimmt den insgesamt ins System eingeprägten Impulsfluss.
		  \item Das System antwortet auf den unphysikalisch erzwungenden Fluss mit einem 'echten' Impulsfluss in die entgegengesetzte Richtung. Dadurch ergeben sich Geschwindigkeitsprofile in der oberen und unteren Hälfte der Simulationsbox.
		  
  		  \item Berechnung der Viskosität
  		  		 \begin{equation}
  		   		 	\eta = \frac{-j_{xz} }{\dot{\gamma} },\quad j_{xz} = \frac{p_{\mathrm{total}}}{2 t L_x L_y},\quad p_{\mathrm{total} }= \sum (p_{x,n_c} - p_{x,n_1})
  		   		\end{equation}
  		   
  		   $j_{xz}$ ... Eingeprägter Impulsfluss\\
  		   $L_x,\,L_y$ ... Längen der Kästchen\\
  		   $t$ ... Zeit\\
  		   $\dot{\gamma} = \frac{\mathrm{d}v_x}{\mathrm{dz}}$ ergibt sich aus der Geschwindigkeitsverteilung
		  
		 \item Setup der LAMMPS MD Simulationen von \cite{Tenney2010} 
		 		\begin{itemize}
		 			\item 3D orthogonale Simulationsbox
		 			\item Periodische RB
		 			\item Verlet Zeitschrittverfahren
		 			\item Equilibrisierung bei \SI{10}{K} mithilfe eines Berendesen Thermostats (kein Impulstausch findet statt)
		 			\item Simulationbox wird in 20 Kästchen entlang der z-Achse eingeteilt
		 			\item Impulstausch zwischen erster und mittlerer Box
		 			\item Tauschfrequenz: Nach jedem 1., 10., 100. Simulationszeitschritt
		 			\item Gesamtzahl der Simulationschritte: $2e6$
		 		\end{itemize}
		 	
		 
		 
		\begin{figure}[H]
	    	\includegraphics[width=0.4\textwidth]{figs/rnemd_box}%
	    	\caption{RNEMD Simulationsbox aus \cite{Tenney2010}, Fig. 1.}
   		\end{figure}		 
		 
	\end{itemize}



\subsection{Gleichgewichts-MD Methoden (equilibrium MD, EMD)}
Viskosität wird über Fluktuationen der \textbf{Nebendiagonalelemente des Spannungstensor} bestimmt. Die erhaltenen Wert gelten nur für \textbf{Newtonsche Fluide} ($\tau \propto \dot{\gamma}$). Prinzipiell sind die Ausdrücke nur im Grenzfall unendlich langer Simulationszeit und unendlich großer Simulationszelle gültig. Zusätzlich liegt typischerweise signifikantes thermisches Rauschen vor, was das SNR verringert. Daher ist eine große Zahl von Simulationsschritten nötig.\cite{Tenney2010}.

	\subsubsection{Autokorrelation des Spannungstensors (Green-Kubo)}
		\begin{itemize}
			 \item Green-Kubo-Relationen (Primärquellen: \cite{Green1954,Kubo1957}, Sekundärquellne für Viskositäts-Formulierung: \cite{Kirova2015,Tenney2010})
			 	\begin{equation}\label{eqn:green_kubo}
			 		\eta = \beta V \int_0^\infty \langle P_{xz}(0) P_{xz}(t) \rangle \mathrm{d} t 
				\end{equation}			 
				
				$\beta = 1/k_\mathrm{B} T$\\
				$V$ ... Volumen der Simulationsbox\\
				$P_{xz}$ ... xz-Komponente des Spannungstensors\\
				$\langle \ldots \rangle$ ... Ensemble-Mittelwert
			 
		 	\item Viskosität wird damit aus der \textbf{Autokorrelationsfunktion} der xz-Komponente der Spannungstensors $P_{xz}$ bestimmt
		 	\item xz-Komponente des Spannungstensors \cite{Kirova2015}
		 		\begin{equation}
					P_{xz} = \sum_{i=1}^N \frac{p_i^x p_i^z}{m_i} - \sum_{i>j}^N (x_i - x_j) \frac{\partial u_{ij} }{\partial z_j}	 	
			 	\end{equation}
			 	
			 	$p_i^{x,z}$ ... Komponente des Impulses des i-ten Teilchens\\ 
			 	$u_{ij}$ ... Interaktionspotential zwischen Teilchen j und Teilchen i\\
			 	$m_i$ ... Masse des i-ten Teilchens\\
			 	$x_i$ ... Ortsvektorkomponente von Teilchen i.
			 	
			\item Für \textbf{isotrope} Systeme existiert eine alternative Formulierung, die die Symmetrie ausnützt\cite{Tenney2010,Daivis1994}.
				\begin{equation}
					\eta = \frac{\beta V}{10} \int_0^\infty \langle P_{ij}^{os}(0) P_{ij}^{os}(t) \rangle \mathrm{d} t
				\end{equation}
			   
			    $P_{ij}^{os}$ ist eine Komponente des spurfreien symmetrischen Anteils des Spannungstensors.
			    \begin{equation}
			    	P_{ij}^{os} = \frac{P_{ij} + P_{ji} }{2} - \delta_{ij} \frac 1 3 \sum_k P_{kk}
			    \end{equation}
			    
    		 \item Setup der LAMMPS MD Simulationen von \cite{Kirova2015,Tenney2010} 
		 		\begin{itemize}
		 			\item 3D orthogonale Simulationsbox
		 			\item Periodische RB
		 			\item Verlet Zeitschrittverfahren
		 			\item Equilibrisierung bei \SI{20}{K} mithilfe eines Berendesen Thermostats (kein Impulstausch findet statt)
		 			\item Methode von \cite{Kirova2015}: MD Trajektorie wird in zeitliche Segmente eingeteilt und zeitliche Mittelung gemäß \refeqn{eqn:green_kubo} wird durch Ensemble-Mittelwert der einzelnen Segmente ersetzt.
		 		\end{itemize}
			 	
		 	\item Bsp in LAMMPS verweist auf \cite{Daivis1994}
		\end{itemize}

	\subsubsection{Einstein Formulierung}
		\begin{itemize}
			 \item Viskosität wird über einen Grenzwert aus der Fluktuation des Spannungstensors ermittelt \cite{Tenney2010}
			 
			 	\begin{equation}
					\eta = \lim_{t\to\infty} \frac{\beta V}{2 t} \left\langle \int_0^t P_{xz}(t')\mathrm{d}t' \right\rangle
				\end{equation}
				
			 \item Kirova \cite{Kirova2015} verweist auf eine andere Formel, die bereits 1960 von Helfand \cite{Helfand1960} beschrieben wurde.
			    \begin{equation}
					\eta = \lim_{N,V,t\to\infty} \frac{\beta}{2 V t} \left\langle \left[\sum_{i=0}^N x_i(t) p_i^z(t) - \sum_{i=1}^N x_i(0) p_i^z(0) \right]^2 \right\rangle
				\end{equation}
				
				Diese hat eine ähnliche Form wie die bekannte Einstein Relation für die Diffusionskonstante
				\begin{equation}
					D = \frac{1}{6t} \left\langle \left[ \sum_{i=0}^N r_i(t) - r_i(0) \right]^2 \right\rangle
				\end{equation}
			 
			 \item Die prinzipielle Vorgangsweise bei der MD Simulation ist dieselbe wie bei der Green-Kubo-Methode.
		\end{itemize}

\section{Experimentelle Befunde}

\subsection{Stress und Viskositätsbetrachtungen}
	\begin{itemize}
		\item Betrachtungen für $\mathrm{SiO_2}$ in \cite{Snoeks2000}
	\end{itemize}



\bibliographystyle{Citation/IEEEbib-c.bst}
\bibliography{Citation/MolecularDynamics}


\end{document}  
